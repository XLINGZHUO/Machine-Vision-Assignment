\subsection{Conventional Image Processing in Apple Counting}
\cite{Qilemuge2018} performed wavelet denoising and median filtering on grayscale images of apples for image pre-processing, used Sobel operator for edge detection and then restored the image according to morphological features, and finally used the watershed algorithm to count apples. This scheme avoids the interference of color on recognition, and realizes the recognition and counting of apples to a certain extent, but the occlusion of leaves, apples and other objects has a greater impact on the accuracy. 

\cite{Guennouni2014} used the cascade object detection algorithm and classified the edge features, line features and centre surround features of the target based on Haar-Like features, and built the solution on a movable device to test it on targets such as hand gestures and faces. The project could achieve relatively fast and accurate target recognition on the mobile platform, but the experimental objects did not show interference such as cross-over, which could not better reduce the interference error caused by the occlusion of target objects. 

For traditional image processing, the factors that impact the precision are the quality and complexity within the image, the method of extracting the target features, and the tuning of the classifier. All of the above solutions perform well when the target object has a small variety of colours and is not occluded. For more demanding environments, they perform poorly.

\subsection{Machine Learning in Apple Counting}
\cite{Rahnemoonfar2017} applied an optimised Inception ResNet architecture and synthetic data training to save the cost of manual annotation. They achieved 91\% accuracy on 100 randomly selected real tomato images and were robust to shadows, leaf occlusion and overlap between tomatoes. However, the model was unable to identify fruits other than red due to the singularity of the colour of the training dataset. 

\cite{Chen2017}s' model based on a fully convolutional network and linear regression solves the problem of \cite{Rahnemoonfar2017}, and their model identifies apples that are similar in colour to the leaves on the tree. The model is similar to a texture-based detector, where interest regions are first recognised and then counted. The detection neural network achieved an average $IoU$ of 0.838, and the $L_2$ parametric loss of the counting neural network was only 10.5. However,  they failed to quantify the reason why the predictions exceeded the ground truth.

The advantage of machine learning lies in the acquisition of data features through multi-layer neural networks. The weights adjusted by back propagation improve the precision of the model. But, it depends on the quality and quantity of the dataset; the more complicated the features of the target, the larger the dataset required and the higher the cost of annotation. 