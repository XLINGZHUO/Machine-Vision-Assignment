\subsection{Background}
The rapid growth of artificial intelligence and the Internet of Things (IoT) has made efficient data analysis as well as fine-grained management feasible~\citep{Liu2021}. Smart agriculture is gradually becoming the mainstream of the agriculture in the new era.

At this stage, agricultural automation is concentrated in two areas, intelligent farming with IoT during nurturing period~\citep{ZamoraIzquierdo2019}, and harvesting with flexible claws~\citep{Li2021}. Nevertheless, interventions in the cultivation process are rarely mentioned~\citep{Debnath2022}. Therefore, we apply machine vision as an aid, combining the power of graphic processing units (GPUs) to participate in decision-making in the cultivation to fill the gap.

Moreover, when the profitability of the agricultural industry declines due to consumer demand, trade competition and natural disasters, creating an incentive to reduce labour supply~\citep{Calvin2010}, which leads to higher labour costs, resulting in more growers towards agricultural automation.  

In view of the above background, we try to implement the morphology and HSV based image processing and the YOLOX-based deep learning method for apple yield estimation in orchards respectively. We also compare them in terms of precision, running time and energy consumption in accordance with the experimental results, and discuss the applicability scenarios of each to inspire agricultural industry participants.


\subsection{Assumptions}
For the detection task in this paper, we assume that:
\begin{itemize}
    \item Factors such as ambient light and obstruction of the detection target impact on accuracy.
    \item The design and tuning of deep learning algorithms govern its performance.
    \item Deep learning is more accurate than traditional image processing, but deep learning requires more computational resources and time.
\end{itemize}

\subsection{Aims \& Objectives}
We plan to achieve the following objectives with morphology and HSV based image processing and the deep learning method based on YOLOX respectively:
\begin{itemize}
    \item Identify the apple pixels in the image and count the number of apples,
    \item Compare the accuracy, running time and computational cost of the two methods to discuss their deployment in reality.
\end{itemize}

\subsection{Challenges}
After reviewing all the images in the dataset, we recognize these challenges:
\begin{itemize}
    \item Variations in the intensity of natural light and the uneven colour of the apples themselves.
    \item Occlusion of apple contours from leaves, branches and other apples.
    \item Loss of highlight detail due to overexposure of the camera.
\end{itemize}

